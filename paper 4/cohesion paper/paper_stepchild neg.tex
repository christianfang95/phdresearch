% options for packages loaded elsewhere
\PassOptionsToPackage{unicode=true}{hyperref}
\PassOptionsToPackage{hyphens}{url}
\PassOptionsToPackage{dvipsnames,svgnames*,x11names*}{xcolor}

% specify apa6 document mode with command to have a default
\newcommand{\pandocDocMode}{jou}
% apa6 mode and class options
\documentclass[\pandocDocMode,longtable,floatsintext]{apa6}

% for mode selection options
\usepackage{ifthen}

\newcommand{\forceLongTablePkg}{}

% setup mode ifs
\newif\ifmanmode
\newif\ifdocmode
\newif\ifjoumode
\ifthenelse{\equal{\string \pandocDocMode}{\string man}}{
    \manmodetrue
}{
    \renewcommand{\forceLongTablePkg}{longtable}
    \ifthenelse{\equal{\string \pandocDocMode}{\string doc}}{
        \docmodetrue
    }{
        \ifthenelse{\equal{\string \pandocDocMode}{\string jou}}{
            \joumodetrue
        }{
% None
}}}


\usepackage{\forceLongTablePkg}

% floatsintext on

% other packages
\usepackage{lmodern}
\usepackage{amsmath,amssymb}
\usepackage{ifxetex,ifluatex}
\usepackage{fixltx2e} % provides \textsubscript

% handle different types of tex engines
% if pdftex
\ifnum 0\ifxetex 1\fi\ifluatex 1\fi=0
  \usepackage[T1]{fontenc}
  \usepackage[utf8]{inputenc}
  \usepackage{textcomp} % provides euro and other symbols
% if luatex or xelatex
\else
  \usepackage{unicode-math}
  \defaultfontfeatures{Ligatures=TeX,Scale=MatchLowercase}
\fi

% other language options
\usepackage[american]{babel}
\usepackage{csquotes}
\usepackage{microtype}

% disable microtype protrusion for tt fonts
\UseMicrotypeSet[protrusion]{basicmath}

\usepackage{graphicx,grffile}
% Scale images if necessary, so that they will not overflow the page
% margins by default, and it is still possible to overwrite the defaults
% using explicit options in \includegraphics[width, height, ...]{}
\makeatletter
\def\maxwidth{\ifdim\Gin@nat@width>\linewidth\linewidth\else\Gin@nat@width\fi}
\def\maxheight{\ifdim\Gin@nat@height>\textheight\textheight\else\Gin@nat@height\fi}
\makeatother
\setkeys{Gin}{width=\maxwidth,height=\maxheight,keepaspectratio}

\usepackage{booktabs}
\usepackage{caption}
\usepackage{subcaption}
\usepackage{xcolor}

% Pandoc stuff
\let\tightlist\relax % empty pandoc tight list command

% Hyperlinks and other metadata in pdf
\usepackage{hyperref}
\hypersetup{
            pdftitle={Parents' Perceptions of Stepfamily Cohesion},
            pdfauthor={Christian Fang \& Anne-Rigt Poortman},
            pdfkeywords={stepfamilies, divorce, cohesion},
            colorlinks=true,
            linkcolor=Blue,
            citecolor=Blue,
            urlcolor=Blue,
            breaklinks=true}
\urlstyle{same} % don't use monospace font for urls





% References
\usepackage[backend=biber,sortcites=true,style=apa,sorting=nyt,isbn=false,url=false]{biblatex}
\DeclareLanguageMapping{american}{american-apa}
\addbibresource{references.bib}

% Title page stuff
\title{Parents' Perceptions of Stepfamily Cohesion}
\shorttitle{~}\author{Christian Fang \& Anne-Rigt Poortman}
\affiliation{Utrecht University}

\note{Last updated: \today}
\authornote{%
\noindent Correspondence:

Christian Fang Department of Sociology P.O. Box 80140 3584 CH Utrecht

Email: c.fang@uu.nl
}
% Abstract page
\abstract{%
TBD
}
\keywords{stepfamilies, divorce, cohesion}

\newcommand{\journalOptions}{
\ifthenelse{\equal{\string \pandocDocMode}{\string jou}}{

% Journal specific commands may go here:
}{}
}

\journalOptions{}


% table caption width
\makeatletter
\newcommand\LastLTentrywidth{1em}
\newlength\longtablewidth
\setlength{\longtablewidth}{1in}
\newcommand\getlongtablewidth{%
\begingroup
\ifcsname LT@\roman{LT@tables}\endcsname
\global\longtablewidth=0pt
\renewcommand\LT@entry[2]{\global\advance\longtablewidth by ##2\relax\gdef\LastLTentrywidth{##2}}%
\@nameuse{LT@\roman{LT@tables}}%
\fi
\endgroup}


%% pandoc-tablenos: environment to disable table caption prefixes
\makeatletter
\newcounter{tableno}
\newenvironment{tablenos:no-prefix-table-caption}{
  \caption@ifcompatibility{}{
    \let\oldthetable\thetable
    \let\oldtheHtable\theHtable
    \renewcommand{\thetable}{tableno:\thetableno}
    \renewcommand{\theHtable}{tableno:\thetableno}
    \stepcounter{tableno}
    \captionsetup{labelformat=empty}
  }
}{
  \caption@ifcompatibility{}{
    \captionsetup{labelformat=default}
    \let\thetable\oldthetable
    \let\theHtable\oldtheHtable
    \addtocounter{table}{-1}
  }
}
\makeatother


% PREAMBLE END
% --------------
% DOCUMENT START

\begin{document}
\maketitle

\hypertarget{introduction}{%
\section{Introduction}\label{introduction}}

High rates of divorce and repartnering mean that postdivorce
stepfamilies are common in modern societies
\autocite{raley_divorce_2020}. When parents repartner after divorce,
they are faced with multiple challenges. For example, parents need to
adjust to their new partner and, at the same time, often feel
responsible for fostering good relationships between their child and
their new partner, who becomes - at least by definition - the child's
stepparent \autocite{jensen2017transitioning}. Such processes of family
reorganization are often complex and difficult, which is why parents
might find it difficult to feel that their stepfamily is cohesive
\autocite{ganong2019stepfathers} \autocite{pink1985problem}
\autocite{waldren1990cohesion}. Stepfamily cohesion refers to an overall
perception of unity, closeness, and meaningful involvement regarding
one's stepfamily
{[}\textcite{komter2006strength}{]}\autocite{waldren1990cohesion}
\autocite{jensen2022associations}. Cohesion is the feeling that one's
stepfamily is a coherent and supportive unit, rather than a disjoint
patchwork \autocite{favez2015coparenting}.

Cohesion is conceptually closely related to family belonging (i.e.,
individuals' feelings that they are part of the family; see
\autocite{king2015adolescents}). Feelings of cohesion are important to
investigate due to their contribution to family functioning and family
members' well-being. For example, divorced parents can profit from a
more cohesive stepfamily in terms of higher perceived well-being and
lower stress levels \autocite{waldren1990cohesion}. This can, in turn,
positively affect their parenting, which can benefit their children
\autocite{king2015adolescents}. Children growing up in more cohesive
stepfamilies have been found to exhibit fewer behavioral problems and
score higher on subjective well-being \autocite{shigeto2014roles}.

The limited literature on stepfamily cohesion has mostly focused on the
consequences of (a lack of) stepfamily cohesion (e.g.,
\autocite{duncan1994effects}, \autocite{hong2015interactive},
\autocite{shigeto2014roles}) rather than the antecedents of perceptions
of stepfamily cohesion. The few studies that do consider factors
contributing to perceptions of cohesion usually only consider the
influences of the relationship qualities between different stepfamily
member (e.g., \autocite{jensen2022associations}). Furthermore, most of
these studies only study perceptions of cohesion cohesion only in the
most common stepfamily type (i.e., resident stepfather families, see
e.g., \autocite{favez2015coparenting}, \autocite{waldren1990cohesion}).

This stepfamily type is, however, becomming less and less the default.
Stepfamilies have considerably diversified in recent decades, for
example in terms of residence arrangements. Nowadays, an increasing
share of parents opt for shared residence arrangements (i.e., joint
physical custody) or (to a lesser extent) sole father residence
\autocite{poortman2017shared}. Consequently, more parents experience
their children living only part-time with them or mostly with their
ex-partners, which can have important ramifications for how cohesive
parents perceive their stepfamily. This picture can become even more
complicated when one considers that in many postdivorce stepfamilies
parents have a shared biological child with the new partner, and their
new partners can also have children from their previous relationship,
who also follow a residence arrangement. Such postdivorce stepfamily
diversity is pivotal to consider as it could reveal stepfamily
constellations that are particularly prone to be considerd as less
cohesive than others, with potentially detrimental consequences for
parents and their children living in those types of stepfamilies.

In this study, we comprehensively investigate parents' perceptions of
stepfamily cohesion in diverse stepfamilies using large-scale survey
data. We, first, consider differences between parents who do and who do
not have a shared biological child with their current partners vis-a-vis
perceptions of cohesion. Second, we consider parents' biological child's
and potential stepchildrens' residence arrangements. For this study, we
used the third wave of the New Families in the Netherlands (NFN) survey,
collected in 2020 (N=3,056). NFN is a longitudinal survey based on a
probability sample of Dutch parents who divorced or separated in
2009/10. Using this data provides the unique opportunity to investigate
parents' feelings of cohesion across a wide range of postdivorce
families, such as those with shared residence arrangements.

\hypertarget{theoretical-background}{%
\section{Theoretical Background}\label{theoretical-background}}

In the following, we outline our theoretical arguments regarding how and
why postdivorce stepfamily structure could influence parents'
perceptions of stepfamily cohesion. We present our arguments from the
vantage point of the so-called ``focal parent'' (i.e., the respondent).
The focal parents are all divorced and have a biological child from
their previous relationship. That child follows a residence arrangement
(sole mother/father residence or shared residence). Subsequently, the
focal parents entered a stable, coresidential relationship (i.e., they
cohabit or a married) and thereby formed a stepfamily.

We start by describing the potential influence that having a shared
biological child has on perceptions of cohesion, before describing
potential group differences between focal parents' biological children's
residence arrangements and residence arrangements of the stepchild
(i.e., children from the current partners' former union).

\hypertarget{having-a-shared-biological-child}{%
\subsection{Having a shared biological
child}\label{having-a-shared-biological-child}}

Upwards trends in remarriage and multipartner fertility imply that many
repartnered parents go on to having a shared biological child with their
current partner \autocite{lappegaard2018intergenerational}. For two main
reasons, having such a shared biological child can increase parents'
perceptions of stepfamily cohesion.

One line of argument is based on the presence of a shared biological
child influencing perceptions of cohesion via relationship quality.
Parents might deliberately seek to have a biological child with their
current partner to improve and stabilize the relationship with him or
her (i.e., the so-called ``cement'' or ``concrete'' baby). While doing
so has been shown to potentially have have negative consequences of
their existing biological childrens' well-being
\autocite{sanner2018half}, studies investigating parents' accounts have
found a positive association between having a shared child and the
quality of the relationship with their new partner
\autocite{ivanova2019cementing}. As relationship satisafction has, in
turn, been demonstrated to increase perceptions of stepfamily cohesion
(\autocite{jensen2022associations} \autocite{king2015adolescents}
\autocite{king2016factors}, one would expect that having a shared
biological child is positively related to perceptions of cohesion.

Another line of argument is based on shared children changing parents'
family values and perceptions of their family in a more systemic way.
The birth of a shared biological child is a subtantial family structure
transition, that prompts all family members to renegotiate role,
boundaries, expectations, shared norms and values, and family rituals
and routines \autocite{coleman2013resilience}. For example, whereas
roles and boundaries in stepfamilies tend to be permeable and - to an
extent - ambiguous \autocite{fine1992perceived}, the birth of a common
child can clarify roles and boundaries as stepfamily members become
biologically related to one another (\autocite{anderson1999sibling}
\autocite{pasley1989boundary}). This is in line with findings from
studies showing that the birth of a common child reduces uncertainty
about one's family (\autocite{downs2004family},
\autocite{friedman1994theory}), leading to more positve evaluations of
ones family environment. Having a shared child might also make shared
norms and values more concrete, as norms towards biologically related
kin are stronger than those towards non-related kin. The birth of a
shared child might also give parents the impression that their family is
now ``real'', as it corresponds more closely to the societal sterotype
of what a family is, i.e., the nuclear family. Therefore, we hypothesize
that:

H1: Parents who have a shared biological child with their partner
perceive higher extents of stepfamily cohesion than those who do not
have a shared child.

\hypertarget{residence-of-focal-parents-biological-child}{%
\subsection{Residence of Focal Parent's Biological
Child}\label{residence-of-focal-parents-biological-child}}

Whereas focal parents' potential children with their current partners
will - by definition - live in their common household, their biological
children with their previous partners and potential stepchildren (so
their current partner's biological children from their previous
relationship) can follow different postdivorce residence arrangements.

\hypertarget{residence-of-focal-parents-biological-child-with-the-previous-partner}{%
\subsubsection{Residence of Focal Parents' Biological Child with the
Previous
Partner}\label{residence-of-focal-parents-biological-child-with-the-previous-partner}}

In the Netherlands, three residence arrangements are common: sole mother
residence (about two thirds), shared residence (i.e., joint physical
custody; about one quarter), and sole father residence
\autocite{poortman2017shared}. Therefore, the focal parent might be a
resident parent (i.e., the biological child resides mostly in his/her
household), a nonresident parent (i.e., the biological child might
reside mostly in the household of the ex-partner), or a part-time
resident parent (i.e., the child lives about half of the time in the
focal parent's and half of the time in the ex-partner's household).

Nonresident parents likely perceive the lowest extent of stepfamily
cohesion. For once, nonresident parents' contact opportunities with
their biological children are inherently limited - nonresident parents
typically see their children only every (other) weekend
\autocite{kelly2007children}. SUch limited contact might mean that
nonresident parents could feel that they are missing out on a
substantial part of their children's lives and feel that they are the
``unimportant parent''
(\autocite{kielty2005mothers},\autocite{stewart1999disneyland}.
Furthermore, they also have fewer opportunities for forming a new
stepfamily comprising their biological child, their current partner, and
themselves due to time constraints or their children resisting such
attempts \autocite{jensen2015perceived}. In sum, due to their child
living mostly outside of their own household, nonresident parents might
in fact consider their stepfamily as factually nonexistent, let alone
cohesive \autocite{kielty2005mothers}.

Part-time resident parents faces unique challenges in creating
stepfamily cohesion. One the one hand, their child resides part-time in
their own household, which gives focal parents greater opportunities to
create a stepfamily, for example by incorporating their children into
new family routines \autocite{bakker2015family}. On the other hand,
shared residence arrangements imply that the stepfamily is constantly in
flux \autocite{carlsund2014swedish}. Parents might experience their
child constantly ``entering and leaving'' their family as stressful, as
they have to regularly switch between a life with and without their
child. Though some parents might appreciate such a clear distinction
between time with and without their children
\autocite{botterman2015social}, for others shared residence might be
challenging \autocite{walper2021shared}. The lack of stability and
temporal compartmentalization of family life might make it difficult for
part-time resident parents to feel a sense of cohesion.

Resident parents' stepfamily situation corresponds most closely to what
societal family values prescibe a family should look like: a couple plus
a child living in the same household. This arrangement, thus,
corresponds most closely to the pre-divorce situation, especially in
terms of contact oppotunities. Resident parents see their children
almost every day and can easily engage in family routines and rituals
with them, such as having shared dinners (\autocite{bakker2015family}
\autocite{waller2014residential}). Taken together, these factors could
give focal parents the feeling that they are a ``real'', cohesive,
stepfamily \autocite{weaver2010caught}. We, thus, hypothesize that:

H2a: Resident focal parents perceive the highest extent of stepfamily
cohesion, followed by part-time resident focal parents, and, lastly,
nonresident focal parents.

\hypertarget{presence-and-residence-of-stepchildren}{%
\subsubsection{Presence and Residence of
Stepchildren}\label{presence-and-residence-of-stepchildren}}

Whether focal parents have stepchildren and in which household such
potential stepchildren live might substantially affect parents'
perceptions of stepfamily cohesion. If the focal parent's current
partner also has a child from a previous relationship, logically, both
the focal parent and the current partner are simultaneously biological
parents and stepparents.

On the one hand, it has been argued that both partners already having a
child creates a ``level playing field'' \autocite{fine1996clarity}. Both
partners know what it means to be a parent, which might reduce frictions
and somewhat clarify the otherwise ambiguous role of the stepparent. On
the other hand, acquiring a stepchild is, for many parents, still an
ambiguous gain \autocite{jensen2021theorizing}, which means that parents
might be unclear about how to relate to their new stepchild and what the
stepchild's place in the family is. Such uncertainty about roles and
boundaries might reduce parents' perceptions that their stepfamily is a
cohesive unit \autocite{downs2004family}. Empirical studies
investigating parents' well-being, furthermore, overall point to having
an additional stepchild being demanding and potentially detrimental to
parents' well-being. For once, establishing and negotiating
relationships with a stepchild is an often long and difficult process
\autocite{ganong1999stepparents}. Studies have shown that many parents
report feeling rejected by their stepchildren, and might suffer from
depressive symptoms as a result (\autocite{ganong2011patterns},
\autocite{shapiro2011parenting}). Feeling rejected by ones stepchild is,
logically, likely also detrimental to perceiving one's stepfamily as
cohesive. Additionally, taking on additional parenting responsibilities
might be stressful and time-consuming \autocite{guzzo2019variation}.
Futhermore, it might be stressful to concurrently perform the role of
the biological parent and stepparent, especially in view of the
stepparent role being up to negotiation with the current partner
\autocite{nomaguchi2020parenthood}. As a result, stepparents might be
likely to experience role and coparenting strains, which may induce
conflicts and disagreements with the current partner. Such conflicts
likely being detrimental to feeling that one's stepfamily is cohesive.

The extent to which such negative consequences of having stepchildren
become salient likely differs by where their stepchild lives. Focal
parents' stepchildren can follow similar residence arrangements as those
of their biological children. Hence, focal parents can have a
nonresident stepchild, a part-time resident stepchild, or a resident
stepchild. Focal parents with residential stepchildren are exposed more
frequently and intensively to their stepchildren than are focal parents
with nonresident or part-time resident stepchildren, which implies that
they might be the ones perceiving the lowest extent to stepfamily
cohesion. Therefore, we hypothesize that:

H2b: Focal parents without stepchildren perceive the highest extent of
stepfamily cohesion, followed by those with nonresident stepchildren,
part-time resident stepchildren and, lastly, resident stepchildren.

\hypertarget{data-and-method}{%
\section{Data and Method}\label{data-and-method}}

I calculated the N (with missing values on the relevant variables
excluded, so this is the most conservative sample size.

\begin{longtable}[]{@{}
  >{\raggedright\arraybackslash}p{(\columnwidth - 8\tabcolsep) * \real{0.7826}}
  >{\raggedright\arraybackslash}p{(\columnwidth - 8\tabcolsep) * \real{0.0870}}
  >{\raggedright\arraybackslash}p{(\columnwidth - 8\tabcolsep) * \real{0.0435}}
  >{\raggedright\arraybackslash}p{(\columnwidth - 8\tabcolsep) * \real{0.0435}}
  >{\raggedright\arraybackslash}p{(\columnwidth - 8\tabcolsep) * \real{0.0435}}@{}}
\toprule()
\begin{minipage}[b]{\linewidth}\raggedright
\textbf{Selection}
\end{minipage} & \begin{minipage}[b]{\linewidth}\raggedright
Sample size
\end{minipage} & \begin{minipage}[b]{\linewidth}\raggedright
\end{minipage} & \begin{minipage}[b]{\linewidth}\raggedright
\end{minipage} & \begin{minipage}[b]{\linewidth}\raggedright
\end{minipage} \\
\midrule()
\endhead
\textbf{N original} & 3056 & & & \\
\textbf{N repartnered (mar/cohab)} & 1456 & & & \\
\textbf{N at least one resident child (complete cases)} & 569 & & & \\
\textbf{at least one resident child and focal child \textless{} 18} &
138 & & & \\
\bottomrule()
\end{longtable}

569 seems OK, but we could consider doing a sensitivity power analysis
to be sure.

\printbibliography[title=References]





\end{document}
