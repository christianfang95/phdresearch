% options for packages loaded elsewhere
\PassOptionsToPackage{unicode=true}{hyperref}
\PassOptionsToPackage{hyphens}{url}
\PassOptionsToPackage{dvipsnames,svgnames*,x11names*}{xcolor}

% specify apa6 document mode with command to have a default
\newcommand{\pandocDocMode}{jou}
% apa6 mode and class options
\documentclass[\pandocDocMode,longtable,floatsintext]{apa6}

% for mode selection options
\usepackage{ifthen}

\newcommand{\forceLongTablePkg}{}

% setup mode ifs
\newif\ifmanmode
\newif\ifdocmode
\newif\ifjoumode
\ifthenelse{\equal{\string \pandocDocMode}{\string man}}{
    \manmodetrue
}{
    \renewcommand{\forceLongTablePkg}{longtable}
    \ifthenelse{\equal{\string \pandocDocMode}{\string doc}}{
        \docmodetrue
    }{
        \ifthenelse{\equal{\string \pandocDocMode}{\string jou}}{
            \joumodetrue
        }{
% None
}}}


\usepackage{\forceLongTablePkg}

% floatsintext on

% other packages
\usepackage{lmodern}
\usepackage{amsmath,amssymb}
\usepackage{ifxetex,ifluatex}
\usepackage{fixltx2e} % provides \textsubscript

% handle different types of tex engines
% if pdftex
\ifnum 0\ifxetex 1\fi\ifluatex 1\fi=0
  \usepackage[T1]{fontenc}
  \usepackage[utf8]{inputenc}
  \usepackage{textcomp} % provides euro and other symbols
% if luatex or xelatex
\else
  \usepackage{unicode-math}
  \defaultfontfeatures{Ligatures=TeX,Scale=MatchLowercase}
\fi

% other language options
\usepackage[american]{babel}
\usepackage{csquotes}
\usepackage{microtype}

% disable microtype protrusion for tt fonts
\UseMicrotypeSet[protrusion]{basicmath}

\usepackage{graphicx,grffile}
% Scale images if necessary, so that they will not overflow the page
% margins by default, and it is still possible to overwrite the defaults
% using explicit options in \includegraphics[width, height, ...]{}
\makeatletter
\def\maxwidth{\ifdim\Gin@nat@width>\linewidth\linewidth\else\Gin@nat@width\fi}
\def\maxheight{\ifdim\Gin@nat@height>\textheight\textheight\else\Gin@nat@height\fi}
\makeatother
\setkeys{Gin}{width=\maxwidth,height=\maxheight,keepaspectratio}

\usepackage{booktabs}
\usepackage{caption}
\usepackage{subcaption}
\usepackage{xcolor}

% Pandoc stuff
\let\tightlist\relax % empty pandoc tight list command

% Hyperlinks and other metadata in pdf
\usepackage{hyperref}
\hypersetup{
            pdftitle={Parents' Perceptions of Stepfamily Cohesion},
            pdfauthor={Christian Fang, Anne-Rigt Poortman},
            pdfkeywords={stepfamilies, divorce, cohesion},
            colorlinks=true,
            linkcolor=Blue,
            citecolor=Blue,
            urlcolor=Blue,
            breaklinks=true}
\urlstyle{same} % don't use monospace font for urls





% References
\usepackage[backend=biber,sortcites=true,style=apa,sorting=nyt,isbn=false,url=false]{biblatex}
\DeclareLanguageMapping{american}{american-apa}
\addbibresource{references.bib}

% Title page stuff
\title{Parents' Perceptions of Stepfamily Cohesion}
\shorttitle{~}\author{Christian Fang, Anne-Rigt Poortman}
\affiliation{Utrecht University}

\note{Last updated: \today}
\authornote{%
\noindent Correspondence:

Christian Fang Department of Sociology P.O. Box 80140 3584 CH Utrecht

Email: c.fang@uu.nl
}
% Abstract page
\abstract{%
TBD
}
\keywords{stepfamilies, divorce, cohesion}

\newcommand{\journalOptions}{
\ifthenelse{\equal{\string \pandocDocMode}{\string jou}}{

% Journal specific commands may go here:
}{}
}

\journalOptions{}


% table caption width
\makeatletter
\newcommand\LastLTentrywidth{1em}
\newlength\longtablewidth
\setlength{\longtablewidth}{1in}
\newcommand\getlongtablewidth{%
\begingroup
\ifcsname LT@\roman{LT@tables}\endcsname
\global\longtablewidth=0pt
\renewcommand\LT@entry[2]{\global\advance\longtablewidth by ##2\relax\gdef\LastLTentrywidth{##2}}%
\@nameuse{LT@\roman{LT@tables}}%
\fi
\endgroup}


%% pandoc-tablenos: environment to disable table caption prefixes
\makeatletter
\newcounter{tableno}
\newenvironment{tablenos:no-prefix-table-caption}{
  \caption@ifcompatibility{}{
    \let\oldthetable\thetable
    \let\oldtheHtable\theHtable
    \renewcommand{\thetable}{tableno:\thetableno}
    \renewcommand{\theHtable}{tableno:\thetableno}
    \stepcounter{tableno}
    \captionsetup{labelformat=empty}
  }
}{
  \caption@ifcompatibility{}{
    \captionsetup{labelformat=default}
    \let\thetable\oldthetable
    \let\theHtable\oldtheHtable
    \addtocounter{table}{-1}
  }
}
\makeatother


% PREAMBLE END
% --------------
% DOCUMENT START

\begin{document}
\maketitle

\hypertarget{introduction}{%
\section{Introduction}\label{introduction}}

Divorce and stepfamily formation are common in modern societies
\autocite{raley_divorce_2020}. When parents repartner, they are faced
with multiple challenges. For example, parents need to learn to live
with a new partner and, at the same time, often feel responsible for
fostering good relationships between their child and their new partner,
who becomes - at least by definition - the child's stepparent
\autocite{jensen2017transitioning}. Such processes of family
reorganization are often complex and difficult, which is why parents
might find it difficult to feel that their stepfamily is cohesive
\autocite[ \textcite{pink1985problem}
\textcite{waldren1990cohesion}]{ganong2019stepfathers}. Stepfamily
cohesion refers to an overall perception of unity, closeness, and
meaningful involvement regarding one's stepfamily \autocite[
\textcite{waldren1990cohesion}
\textcite{jensen2022associations}]{komter2006strength}.

Feelings of cohesion are important to investigate due to their
contribution to family functioning and family members' well-being.
Divorced parents can profit from a more cohesive stepfamily in terms of
higher perceived well-being and lower stress levels
\autocite{waldren1990cohesion}. This can, in turn, positively affect
their parenting, which benefits their children
\autocite{king2015adolescents}. Children growing up in more cohesive
stepfamilies have been found to exhibit fewer behavioral problems and
score higher on subjective well-being \autocite{shigeto2014roles}.

The limited literature on stepfamily cohesion has mostly focused on the
consequences of (a lack of) stepfamily cohesion \autocite[e.g.,][
\textcite{hong2015interactive}
\textcite{shigeto2014roles}]{duncan1994effects}. The few studies that
consider factors contributing to perceptions of cohesion are usually
limited to considering the influences of the relationship qualities
between stepfamily member \autocite[e.g.,][]{jensen2022associations} and
study cohesion only in the most common stepfamily type (i.e., resident
stepfather families, see e.g., \textcite{favez2015coparenting}
\textcite{jensen2022associations}).

The stereotypical resident stepfather family has, however, become less
of a default stepfamily constellation in recent years. Stepfamilies have
considerably diversified, for example in terms of residence arrangements
\autocite{raley_divorce_2020}. Nowadays, an increasing share of parents
opt for shared residence arrangements (i.e., joint physical custody) or
(to a lesser extent) sole father residence. Consequently, more parents
experience their children living half the time or even mostly with their
ex-partners, with structurally reduced access to the child potentially
implying that such stepfamilies are considered less cohesive. This
picture becomes even more complicated when one considers that in many
postdivorce stepfamilies parents have a shared biological child with the
new partner, and their new partners can also have children from their
previous relationship, who also follow similar residence arrangements.
It is vital to consider such postdivorce stepfamily diversity, as doing
so could reveal stepfamily constellations that are particularly prone to
be considerd as less cohesive than others, with potentially detrimental
consequences for parents and their children living in those types of
stepfamilies.

In this study, we comprehensively investigate parents' perceptions of
stepfamily cohesion in diverse stepfamilies. We first, consider how the
extent to which parents consider their stepfamilies as cohesive differ
between three common stepfamily constellations: a) having only a
residential biological child, b) having a residential biological child
and a residential stepchild, and c) having a residential biological
child and a nonresidential stepchild (or vice-versa). Within these three
constellations, we, furthermore, investigate the effect of part-time
residency of at least one of the children. Second, we consider the
effect of parents having a shared biological child with their current
partners (a ``concrete baby'') vis-a-vis cohesion. For this study, we
used the third wave of the New Families in the Netherlands (NFN) survey,
collected in 2020 (N=3,056). NFN is a longitudinal survey based on a
probability sample of Dutch parents who divorced or separated in
2009/10. Using this data provides the unique opportunity to investigate
parents' feelings of cohesion across a wide range of postdivorce
families, such as those with shared residence arrangements.

\hypertarget{theoretical-background}{%
\section{Theoretical Background}\label{theoretical-background}}

We base our theoretical arguments on various factors that might
influence the extent to which different stepfamilies are considered
cohesive: relationship qualities, opportunities for contact and bonding
between stepfamily members, and the continuity of relationships in
different stepfamily configurations.

Our arguments are, for the most part, phrased from the vantage point of
a ``focal parent'' (i.e., the respondent), who has a child from a
previous relationship and is currently in a stable (i.e., cohabitational
or marital) relationship with their current partner. Their current
partner can also have a stepchild, and the focal parent can have a
biological child with their current partner (i.e., a ``concrete baby'').
In other words, the focal parents are part of stepfamilies where there
are one or two stepchildren, and possibly a halfsibling. We are only
considering stepfamilies in which at least one of the children from the
focal parents' or current partner's former relationship is (part)time
resident in focal parents' household. This is because stepfamily
cohesion likely applies less to stepfamily situations without any
resident children.

\hypertarget{biological-childs-and-stepchilds-residence-arrangements}{%
\subsection{Biological child's and stepchild's residence
arrangements}\label{biological-childs-and-stepchilds-residence-arrangements}}

Both the focal parent's existing biological child and their current
partner's potential child from a previous union can either live
(part-time) with the focal parent (i.e., they are residential) or they
live with the respective ex-partner (i.e., from the perspective of the
focal parent, they are nonresidential). Excluding the scenario where
both children are nonresidential, the combinations of both children's
residence arrangements yield three distinct scenarios or household
structures.

The first - and most simple - scenario is that there is only one
residential child (from the focal parent), but no stepchild (i.e., the
current partner does not have a child from their previous union). In
such a stepfamily, it might be comparatively easy to foster stepfamily
cohesion. While establishing relationships between stepparents and
stepchildren is usually a difficult process
\autocite{ganong2017stepfamily}, it might be easier if there is only one
new family member that needs to be integrated (i.e., only the current
partner)\autocite[
\textcite{pylyser2018stepfamilies}]{gennetian2005one}, and especially so
when all stepfamily members reside in the same household
\autocite{arat2022parental}. In that scenario, the stepparent has ample
opportunities for engaging with their stepchild
\autocite{landon2022stop} and focal parents can guide their current
partners in taking up the role of the stepparent \autocite[
\textcite{ceballo2004gaining}]{ganong2011patterns}. Furthermore, focal
parents and their current partners have many opportunities to create a
sense of family belonging and cohesion by performing family routines and
rituals together \autocite{fang2022family}. Naturally, when all
stepfamily members reside in the same household, this can make
differences in norms, values, and habits especially salient
\autocite{landon2022stop}, though coresidence might also allow many
opportunities for working through such conflicts and establishing a
cordial relationship and stepfamily cohesion.

In the second scenario, both the focal parent and their current partner
have a residential biological child from a previous union. Phrased
differently, the household comprises two residential stepchildren. Life
in such a stepfamily might be more complex than in simple stepfamilies
that only comprise one stepchild \autocite{landon2022stop}. For once,
both the focal parent and their current partner need to get to know
their new stepchild and simultaneously adopt the role of a stepparent
\autocite{pylyser2018stepfamilies}. This can be an often long,
difficult, and at times frustrating process, that both (step)parents go
through at the same time, which can cause frictions. Second,
relationships between stepchildren are often frought with conflict and
ambiguity \autocite[ \textcite{sanner2018half}]{landon2022stop}, which
can have a negative impact on focal parents' perceptions of stepfamily
cohesion. On the other hand, as in the first scenario, both children
being residential gives all stepfamily members ample opportunities to
get to know one another and build family routines
\autocite{landon2022stop}, which implies that the often tumulous phase
of stepfamily members could be overcome relatively easily. On the other
hand, there is the potential for frictions among stepfamily members to
persist perpetually, which could imply that - on average - life in such
``blended'' stepfamilies is more challenging and that these stepfamilies
are, as a result, perceived as less cohesive.

The third scenario involves one of the children being residential, and
the other child being nonresidential. In other words, the children's
residence arrangements are assymmetrical. Such asymmetric residence
arrangements might be even more challenging to navigate than the second
scenario outlined above. For once, whereas it is easy to build
stepfamily cohesion through family rituals and routines when children
are residential, this is more difficult when children are spread across
households \autocite{manning2003complexity}. Practicing routines and
rituals becomes more difficult and less-self evident, for example as
more planning is necessary (e.g., when are both children at home?)
\autocite[ \textcite{schlinzig2019between}]{ambert1986being}. This also
implies that, if one of the children is nonresidential, that parent
might to some extent still be involved in their ``old family'', which
can lead to a less clear separation between the ``old'' and the ``new''
family. Family boundaries might, as a result, also be less clear
\autocite{stewart2005boundary}. Assymmetries in residence arrangements
can, furthermore, also evoke feelings of guilt
\autocite{kalmijn2020guilt}. For example, focal parents with a resident
biological child and a nonresident stepchild might feel guilty about
being able to spend so much time with their child while their partner
cannot do the same. Conversely, if the focal parent is the nonresident
parent, the current partner might feel guilty about them not getting to
see their child a lot, which can spill over into focal parents'
assessment of their stepfamily as cohesive. Based on these arguments, we
expect \emph{stepfamilies with a resident child and no stepchild to be
perceived as most cohesive, followed by stepfamilies with two
residential children, and, lastly, stepfamilies with a residential and a
nonresidential child} (H1a).

These three scenarios can be further complicated by taking into account
that either child or both children might follow a shared residence
arrangment (i.e., they are part-time resident). In case of shared
residence, the child in question constantly moves between the household
of the focal parent and that of the (current partner's) ex-partner. This
implies that focal parents' (and their partners') access to the children
in question is limited \autocite{arat2022parental}, compared to if the
children resided full time-in their household, which can make it
difficult to practice family routines and rituals that involve all core
stepfamily members. Additionally, in case of shared residence, parents
might, to an extent, be still entangled in their ``previous'' families
\autocite{emery1994conceptualizing}. Aspects of the child's life -
financial matters, division of holidays etc. - need to be constantly
negotiated with the ex-partner(s). Resultingly, family boundaries can
become unclear to family members \autocite{zartler2021children}, making
it difficult for parents to perceive their stepfamilies as cohesive. We,
therefore hypothesize that \emph{if either or both the biological child
or stepchild are part-time resident, perceptions of cohesion will be
lower} (H1b).

\hypertarget{having-a-shared-biological-child}{%
\section{Having a shared biological
child}\label{having-a-shared-biological-child}}

Besides comprising children from previous unions, many postdivorce
stepfamilies also include a halfsibling, meaning a shared biological
child of the focal parent and the current partner
\autocite{sanner2020shared}. In postdivorce stepfamilies, having a
shared biological child often has a high symbolic value for parents,
above and beyond fulfilling the desire for (further) offspring. For
example, the commitment hypothesis, having a shared biological child is
used to signal commitment to each other: focal parents (and their
ex-partners) might want to have a child together to show to each other
that they are ``serious'' about their new relationship, that they have
moved on from their prior unions, and that they wish to focus on their
new family \autocite{vikat1999stepfamily}. Additionally, per the
uncertainty reduction hypothesis, having a shared child might be a
delibrate strategy to reduce uncertainty about the new relationship
\autocite{uncertainty}. Stepfamilies are considered less
institutionalized than ``first-time, two biological parents families''
\autocite{cherlin1978remarriage}, which might mean that parents feel
ambiguity regarding the roles and boundaries in their stepfamilies. By
reproducing, the stepfamily becomes more alike a first-time family,
which can make roles, boundaries, norms, and values clearer, which can
lead to more positive assessments of stepfamily cohesion. Relatedly, the
birth of a biological child can tightly integrate all family members, as
the birth of a further biological child establishes blood ties between
all stepfamily members \autocite{bernstein1990yours}. For example,
whereas many parents perceive their current partners (i.e., the
stepparent) to be outsiders, having a biological child with them can
integrate them more tightly into the existing stepfamily and make them a
``real'' family member \autocite[
\textcite{castren2015insiders}]{bernstein1990yours}. Parents' children
from their prior relationships become biologically related to the new
child (as halfsiblings), which might lead to closer and more amicable
bonds between them \autocite{sanner2018half}. Thus, the birth of a
shared biological child can evoke in the focal parent the perception
that their stepfamily is now ``complete'' and tight-knit and
``cemented'' - thus, cohesive \autocite{ganong1988mutual}.

On the other hand, there are also reasons to assume that the birth of a
common biological child might not substantially change - or even
decrease - the extent to which stepfamilies are perceived as cohesive.
For once, the birth of a mutual child does not necessarily lead
stepparents to invest more in their stepchildren, which implies that one
of the parents (probably the mother) has to care for both their
stepchild as well as their biological child(ren)
\autocite{stewart2005birth}. This can be stressful and evoke perceptions
of unfairness, leading to reduced perceptions of cohesion. Furthermore,
the birth of a shared child can also induce conflict among stepfamily
members. For example, the existing biological children might be opposed
to the birth of the child, or they might feel ``second class children''
as parental attention and investments might shift towards the newborn
\autocite[ \textcite{baham2008sibling}]{sanner2018half}. Resultingly,
they might feel resentment towards their (step)parents. Focal parents
might pick up on such conflicts, which can reduce their perceptions of
stepfamily cohesion.

Prior research and theoretical arguments thus paint a mixed picture.
Ganong and Coleman concluded that while there is not much evidence that
a concrete baby ``works'' in the way parents might intend, they argued
that a concrete baby might still ``work'' as intended due to cognitive
bias: parents might post hoc argue that ``they had a child, therefore it
was the right decision'' \autocite[
\textcite{ganong1988mutual}]{ganong2017stepfamily}, which aligns with
findings that the birth of a common child leads to more postive
assessements of relationship qualities \autocite{ivanova2019cementing}.
We, thus, assume that \emph{Parents who have a shared biological child
assess their stepfamilies as more cohesive than those who do not have a
biological child} (H2).

\printbibliography[title=References]





\end{document}
