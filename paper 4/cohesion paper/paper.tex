% options for packages loaded elsewhere
\PassOptionsToPackage{unicode=true}{hyperref}
\PassOptionsToPackage{hyphens}{url}
\PassOptionsToPackage{dvipsnames,svgnames*,x11names*}{xcolor}

% specify apa6 document mode with command to have a default
\newcommand{\pandocDocMode}{jou}
% apa6 mode and class options
\documentclass[\pandocDocMode,longtable,floatsintext]{apa6}

% for mode selection options
\usepackage{ifthen}

\newcommand{\forceLongTablePkg}{}

% setup mode ifs
\newif\ifmanmode
\newif\ifdocmode
\newif\ifjoumode
\ifthenelse{\equal{\string \pandocDocMode}{\string man}}{
    \manmodetrue
}{
    \renewcommand{\forceLongTablePkg}{longtable}
    \ifthenelse{\equal{\string \pandocDocMode}{\string doc}}{
        \docmodetrue
    }{
        \ifthenelse{\equal{\string \pandocDocMode}{\string jou}}{
            \joumodetrue
        }{
% None
}}}


\usepackage{\forceLongTablePkg}

% floatsintext on

% other packages
\usepackage{lmodern}
\usepackage{amsmath,amssymb}
\usepackage{ifxetex,ifluatex}
\usepackage{fixltx2e} % provides \textsubscript

% handle different types of tex engines
% if pdftex
\ifnum 0\ifxetex 1\fi\ifluatex 1\fi=0
  \usepackage[T1]{fontenc}
  \usepackage[utf8]{inputenc}
  \usepackage{textcomp} % provides euro and other symbols
% if luatex or xelatex
\else
  \usepackage{unicode-math}
  \defaultfontfeatures{Ligatures=TeX,Scale=MatchLowercase}
\fi

% other language options
\usepackage[american]{babel}
\usepackage{csquotes}
\usepackage{microtype}

% disable microtype protrusion for tt fonts
\UseMicrotypeSet[protrusion]{basicmath}

\usepackage{graphicx,grffile}
% Scale images if necessary, so that they will not overflow the page
% margins by default, and it is still possible to overwrite the defaults
% using explicit options in \includegraphics[width, height, ...]{}
\makeatletter
\def\maxwidth{\ifdim\Gin@nat@width>\linewidth\linewidth\else\Gin@nat@width\fi}
\def\maxheight{\ifdim\Gin@nat@height>\textheight\textheight\else\Gin@nat@height\fi}
\makeatother
\setkeys{Gin}{width=\maxwidth,height=\maxheight,keepaspectratio}

\usepackage{booktabs}
\usepackage{caption}
\usepackage{subcaption}
\usepackage{xcolor}

% Pandoc stuff
\let\tightlist\relax % empty pandoc tight list command

% Hyperlinks and other metadata in pdf
\usepackage{hyperref}
\hypersetup{
            pdftitle={Parents' Perceptions of Stepfamily Cohesion},
            pdfauthor={Christian Fang, Anne-Rigt Poortman},
            pdfkeywords={stepfamilies, divorce, cohesion},
            colorlinks=true,
            linkcolor=Blue,
            citecolor=Blue,
            urlcolor=Blue,
            breaklinks=true}
\urlstyle{same} % don't use monospace font for urls





% References
\usepackage[backend=biber,sortcites=true,style=apa,sorting=nyt,isbn=false,url=false]{biblatex}
\DeclareLanguageMapping{american}{american-apa}
\addbibresource{references.bib}

% Title page stuff
\title{Parents' Perceptions of Stepfamily Cohesion}
\shorttitle{Stepfamily Cohesion}
\author{Christian Fang, Anne-Rigt Poortman}
\affiliation{Utrecht University}

\note{Last updated: \today}
\authornote{%
\noindent Correspondence:

Christian Fang Department of Sociology P.O. Box 80140 3584 CH Utrecht

Email: c.fang@uu.nl
}
% Abstract page
\abstract{%
TBD
}
\keywords{stepfamilies, divorce, cohesion}

\newcommand{\journalOptions}{
\ifthenelse{\equal{\string \pandocDocMode}{\string jou}}{

% Journal specific commands may go here:
}{}
}

\journalOptions{}


% table caption width
\makeatletter
\newcommand\LastLTentrywidth{1em}
\newlength\longtablewidth
\setlength{\longtablewidth}{1in}
\newcommand\getlongtablewidth{%
\begingroup
\ifcsname LT@\roman{LT@tables}\endcsname
\global\longtablewidth=0pt
\renewcommand\LT@entry[2]{\global\advance\longtablewidth by ##2\relax\gdef\LastLTentrywidth{##2}}%
\@nameuse{LT@\roman{LT@tables}}%
\fi
\endgroup}


%% pandoc-tablenos: environment to disable table caption prefixes
\makeatletter
\newcounter{tableno}
\newenvironment{tablenos:no-prefix-table-caption}{
  \caption@ifcompatibility{}{
    \let\oldthetable\thetable
    \let\oldtheHtable\theHtable
    \renewcommand{\thetable}{tableno:\thetableno}
    \renewcommand{\theHtable}{tableno:\thetableno}
    \stepcounter{tableno}
    \captionsetup{labelformat=empty}
  }
}{
  \caption@ifcompatibility{}{
    \captionsetup{labelformat=default}
    \let\thetable\oldthetable
    \let\theHtable\oldtheHtable
    \addtocounter{table}{-1}
  }
}
\makeatother


% PREAMBLE END
% --------------
% DOCUMENT START

\begin{document}
\maketitle

\hypertarget{introduction}{%
\section{Introduction}\label{introduction}}

Divorce and stepfamily formation are common in modern societies
\autocite{raley_divorce_2020}. When parents repartner, they are faced
with multiple challenges. For example, parents need to learn to live
with a new partner and, at the same time, often feel responsible for
fostering good relationships between their child and their new partner,
who becomes - at least by definition - the child's stepparent
\autocite{jensen2017transitioning}. Such processes of family
reorganization are often complex and difficult, which is why parents
might find it difficult to feel that their stepfamily is cohesive
\autocite[ \textcite{pink1985problem}
\textcite{waldren1990cohesion}]{ganong2019stepfathers}. Stepfamily
cohesion refers to an overall perception of unity, closeness, and
meaningful involvement regarding one's stepfamily \autocite[
\textcite{waldren1990cohesion}
\textcite{jensen2022associations}]{komter2006strength}. It is the
feeling that one's stepfamily is a coherent and supportive unit, rather
than a disjoint patchwork \autocite{favez2015coparenting}. Cohesion is
conceptually closely related to family belonging (i.e., individuals'
feelings that they are part of the family; see e.g., King, Boyd, \&
Thorsen, 2015). Feelings of cohesion are important to investigate due to
their contribution to family functioning and family members' well-being.
Divorced parents can profit from a more cohesive stepfamily in terms of
higher perceived well-being and lower stress levels (Waldren et al.,
1990). This can, in turn, positively affect their parenting, which
benefits their children (King et al., 2015). Children growing up in more
cohesive stepfamilies have been found to exhibit fewer behavioral
problems and score higher on subjective well-being (Shigeto,
Mangelsdorf, \& Brown, 2014).

The limited literature on stepfamily cohesion has mostly focused on the
consequences of (a lack of) stepfamily cohesion, (e.g., Duncan, Duncan,
\& Hops, 1994; Hong et al., 2015; Shigeto et al., 2014). The few studies
that consider factors contributing to perceptions of cohesion are
usually limited to considering the influences of the relationship
qualities between stepfamily member
\autocite[e.g.,][]{jensen2022associations}. Besides it seeming somehwat
trite that relationship qualities are related to perceptions of
cohesion, perceptions of cohesion are likely influenced by more than
just relationship qualities. Rather, structural aspects of stepfamilies,
such as whether there are shared biological children resident in the
household, or postdivorce residence arrangements have been shown to
substantially affect parents' perceptions of what their family
constitues in the first place, and how they perceive living in ``their''
stepfamily. Therefore, it is conceivable that there might be systematic
differences between the extent to which stepfamilies are considered
cohesive.

Disregarding stepfamily structure obscures the substantial heterogeneity
among contemporary postdivorce stepfamilies. Whereas in the past, most
stepfamilies were formed after mothers remarried (i.e., most
stepfamilies were stepfather families), at present, postdivorce
stepfamilies are an evermore heterogeneous group. The primary reason for
this growing heterogeneity is the widespread adoption of postdivorce
residence arrangements other than sole mother residence. Nowadays, in
many (Western) countries, it is becoming increasingly common that
fathers are (more) involved with their children after divorce. This
finds expression in more and more parents practicing shared residence
(i.e., shared physical custody). Parents practicing such shared
residence arrangements have been found to experience family life
differently than their sole resident peers, for example in terms of a
sense of preservation of free time. This could lead to parents who
practice shared residence perhaps - on average - perceiveing their
stepfamilies as relatively more cohesive. Due to the connection between
perceptions of cohesion and parent's and children's well-being,
considering differences in cohesion among postdivorce families could
uncover stepfamily constellations that are especially (un)likely to be
considered cohesive, and more targeted interventions could be
constructed for these families in particular.

In this study, we comprehensively investigate parents' perceptions of
stepfamily cohesion in diverse stepfamilies using large-scale
probabilistic survey data. We, first, consider differences between
parents who do and who do not have a shared biological child with their
current partners vis-a-vis perceptions of cohesion. Second, we consider
the of parents' biological child's and potential stepchildrens'
residence arrangements. While not central to our study, our analysis
also controls for other factors which might affect perceptions of
stepfamily cohesion, such as the child's age and gender. For this study,
we used the third wave of the New Families in the Netherlands (NFN)
survey, collected in 2020 (N=3,056). NFN is a longitudinal survey based
on a random probability sample of Dutch parents who divorced or
separated in 2009/10. Using this data provides the unique opportunity to
investigate parents' feelings of cohesion across a wide range of
postdivorce families, such as those with shared residence arrangements.

\hypertarget{theoretical-background}{%
\section{Theoretical Background}\label{theoretical-background}}

In the following, we will outline our theoretical arguments regarding
how and why postdivorce stepfamily structure can influence parents'
perceptions of stepfamily cohesion. We phrase our arguments from the
vantage point of the so-called ``focal parents'' (i.e., the respondents
of the NFN survey). The focal parents are all divorced and have a child
from their previous relationship. Subsequently, they entered a stable,
coresidential relationship (i.e., they cohabit or a married).

We start by describing the potential influence that having a shared
biological child has on perceptions of cohesion, before describing
potential group differences between focal parents' biological children's
residence arrangements and residence arrangements of the stepchild
(i.e., children from the current partners' former union).

\hypertarget{having-a-shared-biological-child}{%
\subsection{Having a shared biological
child}\label{having-a-shared-biological-child}}

Many repartnered parents go on to having a shared biological child with
their current partner, with that new child being the halfsibling of
parents' children from their prior unions. For two main reasons, having
such a shared biological child can increase parents' perceptions of
stepfamily cohesion.

The first argument is based on relationship quality. As has been
frequently stated in form of the ``concrete baby'' hypothesis, parents
might deliberately seek to have a biological child with their current
partner to improve and stabilize the relationship with him or her. While
doing so might have negative consequences of their existing biological
childrens' well-being, empirical assessments of this conjecture have
indeed shown a positive association between having a shared biological
child and relationship satisfaction. Relationship satisafction has, in
turn, been consistently shown to have a positive association with
stepfamily cohesion.

Another argument is based on having a shared child changing parents'
family values and perceptions of their family in a more systemic way.
The birth of a shared biological child represents a subtantial family
structure transition, that requires all family members to renegotiate
role, boundaries, expectations, shared norms and values, and family
rituals and routines: a baby changes everything. Whereas such
transitions can be onerous, they offer countless opportunities for
improving upon the status quo. For example, whereas roles and boundaries
in stepfamilies tend to be permeable and - to an extent - ambiguous,

-\textgreater{} roles need to be renegotiated

-\textgreater{} greater clarity about roles

-\textgreater{} adds stability to union -\textgreater{} more optimistic
assessments -\textgreater{} less uncertainty -\textgreater{} uncertainty
reduction theory {[}uncertainty{]} -\textgreater{} reducing uncertainty
leads to higher family functioning and postivite global assessments of
the family. Studies have found that having a shared child reduces
uncertainty, makes roles less unclear.

Therefore, we hypothesize that:

\begin{verbatim}
H1: Parents who have a shared biological child with their partner perceive higher extents of stepfamily cohesion than those who do not have a shared child.
\end{verbatim}

\hypertarget{childs-postdivorce-residence-arrangement}{%
\subsection{Child's postdivorce residence
arrangement}\label{childs-postdivorce-residence-arrangement}}

-\textgreater{} in NL -\textgreater{} permeability -\textgreater{} in
case of \textbf{shared residence}: time with and time without kids.
-\textgreater{} in case of \textbf{resident children}

\hypertarget{residence-stepchildren}{%
\subsection{residence stepchildren}\label{residence-stepchildren}}

-\textgreater{} having kids per se changes perception of what ones
family is, though not uniformely so: strongly depends on where the child
lives.

\hypertarget{data-and-method}{%
\section{Data and Method}\label{data-and-method}}

-\textgreater{} selection: -\textgreater{} Exclude people who do not
currenlty have a new partner (singletons + LAT relationships), N
remaining=1,465 -\textgreater{} there needs to be at least one part-time
resident child in the household -\textgreater{} only new biochild not
enough -\textgreater{} would probably not be considered a
steprelationship/stepfamily

\hypertarget{measures-of-dependent-variable}{%
\subsection{Measures of Dependent
Variable}\label{measures-of-dependent-variable}}

\emph{Stepfamily cohesion}

\hypertarget{measures-of-independent-variables}{%
\subsection{Measures of Independent
Variables}\label{measures-of-independent-variables}}

\emph{Having a biological child with the current partner}

\emph{Presence stepchild}

\emph{Residence stepchild}

\emph{Residence biological child}

\hypertarget{measures-of-control-variables}{%
\subsection{Measures of Control
Variables}\label{measures-of-control-variables}}

\hypertarget{analytical-strategy}{%
\subsection{Analytical Strategy}\label{analytical-strategy}}

First, we imputed missing values using multiple imputations by chained
equations in R.

Next, we estimated two multilevel linear regression models using lme4 in
R. We used multilevel -\textgreater{} Multiple regression in Python

\hypertarget{results}{%
\section{Results}\label{results}}

\hypertarget{discussion-and-conclusion}{%
\section{Discussion and Conclusion}\label{discussion-and-conclusion}}

\printbibliography[title=References]





\end{document}
